\documentclass[11pt,letterpaper, oneside, headings=normal,final]{article}


\usepackage[paper=a4paper,left=2cm,right=2cm,top=2.5cm,bottom=2cm]{geometry}
\usepackage[english,ngerman]{babel}
\usepackage[T1]{fontenc}
\usepackage[latin1]{inputenc}
\usepackage{verbatim}

\usepackage{ifthen}
\usepackage{fancyhdr}
\usepackage{color}

\usepackage{placeins}
\usepackage{amssymb}
\usepackage{amsmath} 
\usepackage{graphicx}

\usepackage[colorlinks=true, allcolors=blue]{hyperref}
\usepackage[figure,table]{hypcap}

\usepackage{pgf,tikz}
\usepackage{mathrsfs}
\usepackage{setspace}
\usetikzlibrary[patterns]
\usepackage{pdfpages}

\usepackage{multirow}
%%%%%%%%%%%%%%%%%%%%%%%%%%%%%%%%%%%%%%%%%%%%%%%%%%%%%
% Farben f�r Hyperlinks z.B.
\definecolor{lt_gray}{rgb}{.8,.8,.8}
\definecolor{red}{rgb}{.90196,.0,.2}
%\definecolor{green2}{rgb}{0,.5,.0}
%\definecolor{myblue}{rgb}{.27,.51,.71} % 69 130 181
\definecolor{myblue}{rgb}{0,.5,.0}
\definecolor{green2}{rgb}{.27,.51,.71}
%
\newcommand{\Th}{\mathcal{T}_h}
\newcommand{\OmegaBg}{\widehat{\Omega}}
\newcommand{\ThBg}{\widehat{\mathcal{T}_h}}
%%%%%%%%%%%%%%%%%%%%%%%%%%%%%%%%%%%%%%%%%%%%%%%%%%%%%
\hypersetup{pdftitle={},%
pdfauthor={Florian Streitb�rger},%
colorlinks=true,%
linkcolor={myblue},%
anchorcolor={black},%
citecolor={green2},%
filecolor={magenta},%
menucolor={red},%
urlcolor={cyan},%
pdfstartview={Fit},
pdfpagemode={UseOutlines},
draft=true}
%%%%%%%%%%%%%%%%%%%%%%%%%%%%%%%%%%%%%%%%%%%%%%%%%%%%%
% Abst�nde
\parskip1ex plus.2ex minus.2ex    % Abstand zwischen Abs"atzen: ca. H"ohe von "x"
\renewcommand{\baselinestretch}{0.95} % Eineinzwanzigstelzeilig
\frenchspacing                  % Kein Mehrabstand nach Satzende
\setcounter{secnumdepth}{4} % Durchnumerieren bis subsubsection
\setcounter{tocdepth}{4} % Durchnumerieren bis subsubsection
%%%%%%%%%%%%%%%%%%%%%%%%%%%%%%%%%%%%%%%%%%%%%%%%%%%%%
%%%%%%%%%%%%%%%%%%%%%%%%%%%%%%%%%%%%%%%%%%%%%%%%%%%%%%
% Einr�ckTiefe der ersten Zeile f�r alle folgenden Abs�tze
\parindent0cm
%%%%%%%%%%%%%%%%%%%%%%%%%%%%%%%%%%%%%%%%%%%%%%%%%%%%%%

\usepackage{lastpage}
\usepackage{colortbl}
\usepackage{arydshln}

\usepackage{fancyhdr}
\pagestyle{fancy}
\fancyhf{}
\renewcommand\headrulewidth{0pt}
\cfoot{\thepage{} von \pageref{LastPage}}

\definecolor{light-gray}{gray}{0.95}

\renewcommand{\theequation}{\arabic{section}.\arabic{equation}}

\begin{document}
{\begin{center}
\begin{tabular}{c}
{\huge \textbf{Jahresbericht}}\\
{\huge \textbf{im strukturierten Promotionsstudiengang}}
\end{tabular}
\\[0.2cm]
\begin{tabular}{c}
{\large Dritter Bericht (15.01.21 - 14.01.22)}\\[0.2cm]
vorgelegt von Florian Streitb�rger
\end{tabular}
\\[0.2cm]

\ \\
\begin{tabular}{ll}
Mat.Nr.: 165759,& E-Mail: \href{mailto:florian.streitbuerger@math.tu-dortmund.de}{\nolinkurl{{florian.streitbuerger@math.tu-dortmund.de}}}\\			 
\end{tabular}
\end{center}}
\vspace{1cm}
\section{Forschung}
Im dritten Jahr des strukturierten Promotionsstudiengangs unter der Betreuung von JProf. Dr. Sandra May haben wir die DoD-Stabilisierung von der linearen Advektionsgleichung auf Systeme von nicht-linearen Erhaltungsgleichungen in einer Dimension erweitert. Bei der DoD-Stabilisierung handelt es sich um eine Stabilisierung in Form von sogenannten Bestrafungs-Termen (\textit{engl. Penalty terms}), die ein stabiles Update von hyperbolischen Erhaltungsgleichungen auf Cut Cell Gittern unter Verwendung von unstetigen Galerkin-Verfahren m�glich machen. 

Zu Beginn des dritten Promotionsjahres haben wir zun�chst daran gearbeitet den L$^2$ Stabilit�tsbeweis auf nicht-lineare skalare Erhaltungsgleichungen zu erweitern. Hierbei war es hilfreich die beiden Flussrichtungen zun�chst getrennt voneinander zu betrachten, bevor wir schlie�lich dazu in der Lage waren die Stabilisierung geeignet anzupassen, sodass wir den Beweis f�r den allgemeinen Fall f�hren konnten.

Zeitgleich haben wir die Stabilisierung von skalaren Erhaltungleichungen auf Systeme von linearen Erhaltungsgleichungen erweitert. Dabei haben wir ausgenutzt, dass die Jacobi-Matrix des Flusses bei Systemen von hyperbolischen Gleichungen stets diagonalisierbar ist. So konnten wir das lineare System mithilfe einer Transformation auf den Fall von ungekoppelten skalaren Erhaltungsgleichungen zur�ckf�hren. Mit der dazu passenden R�cktransformation war es m�glich Stabilit�tsterme f�r Systeme von linearen Erhaltungsgleichungen zu finden. 

Anschlie�end haben wir die beiden Projekte zusammengef�hrt und uns Systeme nichtlinearer Erhaltungsgleichungen am Beispiel der inkompressiblen Euler-Gleichungen in 1D angeguckt. Unsere zahlreichen numerischen Tests zeigen, dass die DoD-Stabilisierung auch in diesem Setting funktioniert und ein stabiles Verhalten und die richtigen Konvergenzordnungen zeigt. In letzter Konsequenz haben wir die Ergebnisse zusammengefasst und im Laufe des Jahres in \cite{AMC2021} final ver�ffentlicht.

Im Verlauf des Jahres haben wir uns noch mit weiteren Fragestellungen bez�glich der DoD-Stabilisierung und Cut Cell Gittern im Allgemeinen besch�ftigt: Zum einen haben wir uns damit befasst inwiefern sich die Stabilisierung �ndert, wenn bei einem Cut Cell Gitter mehrere kleine Cut Cells hintereinander in Flussrichtung auftreten und die Ausfluss-Zelle einer kleinen Cut Cell nicht gro� genug ist, um die verteilte Masse aufzunehmen. Hierbei konnten wir erste Erfolge verbuchen und eine Formulierung f�r die Stabilisierung bei Verwendung von st�ckweise konstanten Polynomen. Bei st�ckweise Polynomen h�heren Grades ist dieser Forschungsabschnitt allerdings noch in Arbeit. 

Zum Ende des Jahres haben wir unseren Fokus auf die Entwicklung der Stabilisierung in 2D gelegt. Daraufhin wurde der Code und die Stabilisierung so umgeschrieben, dass auch Tests mit st�ckweisen Polynomen h�herer Ordnung bei linearen Problemen m�glich sind. Die numerischen Konvergenztests zeigen die zu erwartenden Konvergenzordnungen in diesen F�llen: In der L$^1$ Norm zeigen die Fehler die volle Ordnung, wohingegen bei der L$^\infty$ Norm die Fehlerraten vor allem f�r niedrige Polynomgrade leicht reduziert sind. 

Au�erdem waren wir in der Lage das L$^2$ Stabilit�tsresultat aus dem eindimensionalen Fall ins Zweidimensionale zu �bertragen. Das dabei entstandene theoretische Resultat haben wir zusammen mit einigen numerischen Resultaten zusammengeschrieben und eingereicht. Hier steht die Review noch aus.

F�r das kommende Jahr liegt unser Fokus auf nicht-linearen Gleichungen in zwei Dimensionen. Unser gro�es Ziel sind nach wie vor die inkompressiblen Euler-Gleichungen in zwei Dimensionen. 


\begin{thebibliography}{99.}
\vspace*{-0.05cm}
    \bibitem{AMC2021}
      S. May, F. Streitb\"urger
      \newblock DoD Stabilization for non-linear hyperbolic conservation laws on cut cell meshes in one dimension. 
      \newblock Appl. Math. Comput. 419, 2022.
\end{thebibliography}

\newpage
\section{Leistungen im Promotionsstudiengang}

\begin{tabular}{l|l|l|l|l}
 \hline 
    \rowcolor{gray!30}
 & \textbf{Semester/}  &  &  \textbf{Name des} &  \\
    \rowcolor{gray!30}    \multirow{-2}{*}{\textbf{Leistung}} & \textbf{Jahr} & \multirow{-2}{*}{\textbf{Anl.}} & \textbf{Veranstalters} & \multirow{-2}{*}{\textbf{Unterschrift}} \\
\hline \hline
    \rowcolor{light-gray}  \multicolumn{5}{c}{\textbf{Promotionsnahe Leistungen}} \\
\hline \hline
& & & & \\
\multirow{-2}{*}{\textit{Teilnahme Konferenz:} SIAM Conference} & && &\\
\multirow{-2}{*}{ on Computational Science} & \multirow{-2}{*}{M�rz 21} & \multirow{-2}{*}{} & \multirow{-2}{*}{-} & \\
\hdashline
& & & & \\
\multirow{-2}{*}{\textit{Teilnahme:} Oberseminar LSIII}  & \multirow{-2}{*}{SS 21} & \multirow{-2}{*}{} & \multirow{-2}{*}{-} & \\
\hdashline
& & & & \\
\multirow{-2}{*}{\textit{Pr�sentation:} ICOSAHOM 2020} \multirow{-2}{*}{} & \multirow{-2}{*}{Juli 21} & \multirow{-2}{*}{} & \multirow{-2}{*}{-} & \\
\hdashline
& & & & \\
\multirow{-2}{*}{\textit{Pr�sentation:} Hirschegg Workshop} & && &\\
\multirow{-2}{*}{On Conservation Laws} & \multirow{-2}{*}{September 21} & \multirow{-2}{*}{} & \multirow{-2}{*}{-} & \\
\hdashline
& & & & \\
\multirow{-2}{*}{\textit{Publikation:} DoD Stabilization for} & && &\\
\multirow{-2}{*}{ non-linear hyperbolic conservation laws} &  \multirow{-2}{*}{Dezember 21} & & \multirow{-2}{*}{-} &\\
\multirow{-2}{*}{ on cut cell meshes in one dimension} & & \multirow{-2}{*}{ } &  & \\
    \rowcolor{light-gray}  \multicolumn{5}{c}{\textbf{Leistungen	wissenschaftlicher	Weiterbildung}} \\
\hline \hline
& & & & \\
\multirow{-2}{*}{\textit{Teilnahme:} Limiter-Techniken f�r} & && &\\
\multirow{-2}{*}{ numerische Verfahren hoher Ordnung} & \multirow{-2}{*}{WS 20/21} & \multirow{-2}{*}{} & \multirow{-2}{*}{D. Kuzmin} & \\
    \rowcolor{light-gray}   \multicolumn{5}{c}{\textbf{Erwerb	�berfachlicher Kompentenzen}} \\
\hline \hline
 & & & & \\
\multirow{-2}{*}{\textit{Tutor:} Neuronale Netze f�r (hyp.)} & \multirow{-2}{*}{} & \multirow{-2}{*}{ } & \multirow{-2}{*}{} & \\ 
\multirow{-2}{*}{ partielle Differentialgleichungen} & \multirow{-2}{*}{SS 21} & \multirow{-2}{*}{ } & \multirow{-2}{*}{S. May} & \\ 
\hdashline
 & & & & \\
\multirow{-2}{*}{\textit{Tutor:} COP-Kurs} & \multirow{-2}{*}{SS 21} & \multirow{-2}{*}{ } & \multirow{-2}{*}{S. May} & \\ 
\hline
\end{tabular}

\vfill
\hrulefill\hfill\hrulefill\\
\hspace*{0.8cm}(JProf. Dr. Sandra May)\hfill(Florian Streitb\"urger)\hspace{1.6cm} 



\end{document}

