\documentclass[11pt,letterpaper, oneside, headings=normal,final]{article}


\usepackage[paper=a4paper,left=2cm,right=2cm,top=2.5cm,bottom=2cm]{geometry}
\usepackage[english,ngerman]{babel}
\usepackage[T1]{fontenc}
\usepackage[latin1]{inputenc}
\usepackage{verbatim}

\usepackage{ifthen}
\usepackage{fancyhdr}
\usepackage{color}

\usepackage{placeins}
\usepackage{amssymb}
\usepackage{amsmath} 
\usepackage{graphicx}

\usepackage[colorlinks=true, allcolors=blue]{hyperref}
\usepackage[figure,table]{hypcap}

\usepackage{pgf,tikz}
\usepackage{mathrsfs}
\usepackage{setspace}
\usetikzlibrary[patterns]
\usepackage{pdfpages}

\usepackage{multirow}
%%%%%%%%%%%%%%%%%%%%%%%%%%%%%%%%%%%%%%%%%%%%%%%%%%%%%
% Farben f�r Hyperlinks z.B.
\definecolor{lt_gray}{rgb}{.8,.8,.8}
\definecolor{red}{rgb}{.90196,.0,.2}
%\definecolor{green2}{rgb}{0,.5,.0}
%\definecolor{myblue}{rgb}{.27,.51,.71} % 69 130 181
\definecolor{myblue}{rgb}{0,.5,.0}
\definecolor{green2}{rgb}{.27,.51,.71}
%
\newcommand{\Th}{\mathcal{T}_h}
\newcommand{\OmegaBg}{\widehat{\Omega}}
\newcommand{\ThBg}{\widehat{\mathcal{T}_h}}
%%%%%%%%%%%%%%%%%%%%%%%%%%%%%%%%%%%%%%%%%%%%%%%%%%%%%
\hypersetup{pdftitle={},%
pdfauthor={Florian Streitb�rger},%
colorlinks=true,%
linkcolor={myblue},%
anchorcolor={black},%
citecolor={green2},%
filecolor={magenta},%
menucolor={red},%
urlcolor={cyan},%
pdfstartview={Fit},
pdfpagemode={UseOutlines},
draft=true}
%%%%%%%%%%%%%%%%%%%%%%%%%%%%%%%%%%%%%%%%%%%%%%%%%%%%%
% Abst�nde
\parskip1ex plus.2ex minus.2ex    % Abstand zwischen Abs"atzen: ca. H"ohe von "x"
\renewcommand{\baselinestretch}{0.95} % Eineinzwanzigstelzeilig
\frenchspacing                  % Kein Mehrabstand nach Satzende
\setcounter{secnumdepth}{4} % Durchnumerieren bis subsubsection
\setcounter{tocdepth}{4} % Durchnumerieren bis subsubsection
%%%%%%%%%%%%%%%%%%%%%%%%%%%%%%%%%%%%%%%%%%%%%%%%%%%%%
%%%%%%%%%%%%%%%%%%%%%%%%%%%%%%%%%%%%%%%%%%%%%%%%%%%%%%
% Einr�ckTiefe der ersten Zeile f�r alle folgenden Abs�tze
\parindent0cm
%%%%%%%%%%%%%%%%%%%%%%%%%%%%%%%%%%%%%%%%%%%%%%%%%%%%%%

\usepackage{lastpage}
\usepackage{colortbl}
\usepackage{arydshln}

\usepackage{fancyhdr}
\pagestyle{fancy}
\fancyhf{}
\renewcommand\headrulewidth{0pt}
\cfoot{\thepage{} von \pageref{LastPage}}

\definecolor{light-gray}{gray}{0.95}

\renewcommand{\theequation}{\arabic{section}.\arabic{equation}}

\begin{document}
{\begin{center}
\begin{tabular}{c}
{\huge \textbf{Jahresbericht}}\\
{\huge \textbf{im strukturierten Promotionsstudiengang}}
\end{tabular}
\\[0.2cm]
\begin{tabular}{c}
{\large Dritter Bericht (15.01.21 - 14.01.22)}\\[0.2cm]
vorgelegt von Florian Streitb�rger
\end{tabular}
\\[0.2cm]

\ \\
\begin{tabular}{ll}
Mat.Nr.: 165759,& E-Mail: \href{mailto:florian.streitbuerger@math.tu-dortmund.de}{\nolinkurl{{florian.streitbuerger@math.tu-dortmund.de}}}\\			 
\end{tabular}
\end{center}}
\vspace{1cm}
\section{Forschung}
\begin{itemize}
\item Zun�chst L2 Stabilit�t f�r nicht-lineare skalare Erhaltungsgleichungen bewiesen. 
\item Parallel an der Erweiterung f�r Systeme gearbeitet. Zun�chst an linearen Systemen getestet + Wellengleichung
\item Anschlie�end erweitert auf Euler-Gleichungen -> Numerische Ergebnisse sehen gut aus. Allerdings noch keine theoretischen Resultate. Hier w�re entropie stabilit�t interessant
\item Ergebnisse zusammengetragen in Paper 
\item Damit befasst mehrere kleine Zellen hintereinander in 1D
\item Weitere Projekte: Limiter auf Cut Cells -> Forschung hier noch nicht sehr weiter. Es gibt keine Limiter h�herer Ordnung auf Cut Cells. 
\item Weiterentwicklung der Methode in 2D. Lineare Advektion f�r h�here Polynomgrade scheint zu funktionieren -> Nur erste ergebnisse..
\end{itemize}

\begin{thebibliography}{99.}
\bibitem{paper}      
C. Engwer, S. May, C. N\"u\ss ing, and F. Streitb\"urger,
      \newblock A stabilized discontinuous Galerkin cut cell method for discretizing the linear transport equation.
      \newblock arXiv:1906.05642,
      \newblock (2019)
\bibitem{proceeding}      
F. Streitb\"urger, C. Engwer, S. May, and C. N\"u\ss ing,
      \newblock Monotonicity considerations for stabilized DG cut cell schemes for the unsteady advection equation
      \newblock arXiv:1912.11933,
      \newblock (2019)
\end{thebibliography}

\newpage
\section{Leistungen im Promotionsstudiengang}

\begin{tabular}{l|l|l|l|l}
 \hline 
    \rowcolor{gray!30}
 & \textbf{Semester/}  &  &  \textbf{Name des} &  \\
    \rowcolor{gray!30}    \multirow{-2}{*}{\textbf{Leistung}} & \textbf{Jahr} & \multirow{-2}{*}{\textbf{Anl.}} & \textbf{Veranstalters} & \multirow{-2}{*}{\textbf{Unterschrift}} \\
\hline \hline
    \rowcolor{light-gray}  \multicolumn{5}{c}{\textbf{Promotionsnahe Leistungen}} \\
\hline \hline
& & & & \\
\multirow{-2}{*}{\textit{Teilnahme Konferenz:} SIAM Conference} & && &\\
\multirow{-2}{*}{ on Computational Science} & \multirow{-2}{*}{M�rz 21} & \multirow{-2}{*}{} & \multirow{-2}{*}{-} & \\
\hdashline
& & & & \\
\multirow{-2}{*}{\textit{Teilnahme:} Oberseminar LSIII}  & \multirow{-2}{*}{SS 21} & \multirow{-2}{*}{} & \multirow{-2}{*}{-} & \\
\hdashline
& & & & \\
\multirow{-2}{*}{\textit{Pr�sentation:} ICOSAHOM 2020} \multirow{-2}{*}{} & \multirow{-2}{*}{Juli 21} & \multirow{-2}{*}{} & \multirow{-2}{*}{-} & \\
\hdashline
& & & & \\
\multirow{-2}{*}{\textit{Pr�sentation:} Hirschegg Workshop} & && &\\
\multirow{-2}{*}{On Conservation Laws} & \multirow{-2}{*}{September 21} & \multirow{-2}{*}{} & \multirow{-2}{*}{-} & \\
\hdashline
& & & & \\
\multirow{-2}{*}{\textit{Publikation:} DoD Stabilization for} & && &\\
\multirow{-2}{*}{ non-linear hyperbolic conservation laws} &  \multirow{-2}{*}{Dezember 21} & & \multirow{-2}{*}{-} &\\
\multirow{-2}{*}{ on cut cell meshes in one dimension} & & \multirow{-2}{*}{ } &  & \\
    \rowcolor{light-gray}  \multicolumn{5}{c}{\textbf{Leistungen	wissenschaftlicher	Weiterbildung}} \\
\hline \hline
& & & & \\
\multirow{-2}{*}{\textit{Teilnahme:} Limiter-Techniken f�r} & && &\\
\multirow{-2}{*}{ numerische Verfahren hoher Ordnung} & \multirow{-2}{*}{WS 20/21} & \multirow{-2}{*}{} & \multirow{-2}{*}{D. Kuzmin} & \\
\hdashline
    \rowcolor{light-gray}   \multicolumn{5}{c}{\textbf{Erwerb	�berfachlicher Kompentenzen}} \\
\hline \hline
 & & & & \\
\multirow{-2}{*}{\textit{Tutor:} Neuronale Netze f�r (hyp.)} & \multirow{-2}{*}{} & \multirow{-2}{*}{ } & \multirow{-2}{*}{} & \\ 
\multirow{-2}{*}{ partielle Differentialgleichungen} & \multirow{-2}{*}{SS 21} & \multirow{-2}{*}{ } & \multirow{-2}{*}{S. May} & \\ 
\hdashline
 & & & & \\
\multirow{-2}{*}{\textit{Tutor:} COP-Kurs} & \multirow{-2}{*}{SS 21} & \multirow{-2}{*}{ } & \multirow{-2}{*}{S. May} & \\ 
\hline
\end{tabular}

\vfill
\hrulefill\hfill\hrulefill\\
\hspace*{0.8cm}(JProf. Dr. Sandra May)\hfill(Florian Streitb\"urger)\hspace{1.6cm} 

\end{document}

